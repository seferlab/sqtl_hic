\documentclass[11pt]{article}
\usepackage[utf8]{inputenc}
\usepackage{url}
\usepackage{chapterbib}
\usepackage[sectionbib]{natbib}
\usepackage[utf8]{inputenc}
\usepackage[T1]{fontenc}

%\usepackage{biblatex}
%\addbibresource{projects.bib}

%\title{\vspace{-4cm} Proximity of sQTLs and target genes in $3$D Genome}
\title{\vspace{-4cm} Hi-C Reveals The Proximity of sQTLs and Their Target Genes in $3$D Genome of Cancer Cells}
\author{} %Emre Sefer}
\date{}
%Joint analyses of multi-tissue Hi-C and eQTL data demonstrate close spatial proximity
%between eQTLs and their target genes

\begin{document}

\maketitle

\section{Abstract}


\section{Introduction}

% SNP introduction
Single nucleotide polymorphisms (SNPs) are the most frequent genetic variants in humans and represent a valuable
resource for investigating the genetic basis of diseases~\cite{shastry2009}. Genome-wide association studies (GWAS) have found
abundant SNPs associated with various traits and diseases, but most of
risk loci lack clear molecular mechanisms~\cite{buniello2018, chang2018}.
Expression quantitative trait locus (eQTL) studies have been employed to identify SNPs that may influence the expression
levels of genes, thereby contributing to the phenotype outcome~\cite{guo2018,zou2018,gong2018}. However, only a moderate proportion
of GWAS-identified loci are strong eQTLs~\cite{westra2013}, which might be partly due to the small sample sizes, the tissues studied,
and a focus on overall gene level expression measurements without consideration of transcript isoforms~\cite{zhang2015}.

% alternative splicing
Alternative splicing (AS) is a molecular mechanism that produces multiple distinct transcript isoforms from a single gene, increasing
the structural transcript variation and proteome diversity. The number of different transcripts in the human transcriptome vastly
exceeds the number of protein-coding genes~\cite{djebali2012}. On average, there are seven different transcript variants encoded by each
protein-coding gene~\cite{harrow2012}. The invention of RNA sequencing
greatly facilitated the identification of AS at the genomic level~\cite{hyung2017}. In human, alternative
splicing can occur in more than $90\%$ of genes in a cell type-, condition- or species-specific manner, which is thought to extensively increase the
number of proteins over the number of genes in a genome~\cite{wang2008, barbosa2012}.

% alternative splicing
Increasing evidence has demonstrated that Alternative Splicing~(AS) events could undergo modulation by genetic variants. The
identification of splicing quantitative trait loci (sQTLs), genetic variants that affect AS events, might represent an important step
toward fully understanding the contribution of genetic variants in disease development. AS process is often altered in cancer cells to
produce aberrant proteins that drive the progression of cancer. A better understanding of the misregulation of alternative splicing will
shed light on the development of novel targets for pharmacological interventions of cancer.

%cancer
In cancer, aberrant splicing patterns are frequently observed in cancer initiation,
progress, prognosis and therapy and known to contribute to carcinogenesis, dedifferentiation and
metastasis~\cite{sveen2015, wang2018}. RNA sequencing have previously
identified a number of cancer-specific transcript isoforms~\cite{hoyos2019}. For example, an alternatively spliced
transcript isoform of the gene encoding spleen tyrosine kinase is frequently
expressed in breast cancer cells but never in matched normal
tissues~\cite{wang2003}.

In cancer, the splicing process is commonly disrupted, resulting in both functional and non-functional end-products. Cancer-specific splicing events are known to contribute to disease progression; however, the dysregulated splicing patterns found on a genome-wide scale have until recently been less well-studied. 

Available evidence reveals that at least $20\%$ of disease-causing single base-pair mutations~(SNPs) affect
splicing~\cite{faustino2003}. Common genetic variation that affects splicing
regulation, referred to as splicing quantitative trait loci~(sQTLs),
can lead to differences in alternative splicing between individuals,
consequently influence disease susceptibility and drug response~\cite{lalonde2011}. Thus, the identification of sQTLs, especially in cancer tissues, might represent an important step toward fully understanding the contribution of genetic variants in tumorigenesis and development.



%SNPs
Single nucleotide polymorphisms (SNPs) are the most frequent genetic variants in humans and represent an invaluable resource to understand
the genetic basis of diseases~\cite{shastry2009}. Genome-wide association studies (GWAS) have found abundant SNPs associated with
various traits, cancers, and diseases, but most of risk loci lack
clear molecular mechanisms~\cite{chang2018}. Genome-wide sQTL mapping
has been achieved for multiple species ranging from human to Arabidopsis Thaliana~\cite{khokhar2019}.
% Expression quantitative trait locus (eQTL) studies have been employed to identify SNPs that may influence
% the expression levels of genes, thereby contributing to the phenotype
% outcome~\cite{guo2017,danyi2018,gong2017}. However, only a moderate proportion of GWAS-identified
% loci are strong eQTLs~\cite{}, which might be partly due to the small
% sample sizes, the tissues studied, and a focus on overall gene level
% expression measurements without consideration of transcript isoforms~\cite{}.

%
Gene regulation and alternative splicing~(AS) are important for cells
and tissues to function. It has been studied from two independent aspects at
the genomic level, the identification of splicing quantitative trait loci (sQTLs) and identification of long-range
chromatin interactions. It is important to understand their relationship, such as whether sQTLs regulate alternative splicing of
genes through physical chromatin interaction. 

We are interested in understanding whether sQTLs regulate their target
genes through physical chromatin interactions. Our data will possibly demonstrate the close spatial proximity between sQTLs and their target genes
among multiple human primary tissues, cancer tissues, and cell lines.

%previous work
This has been previously shown in eQTLs. Chromatin interactions have been shown to be
one of the main mechanisms underlying eQTLs both in cell lines and primary tissues.

%Characterizing the potential functional roles of SNPs that control transcript isoforms
%in human cancer

Although chromatin interactions have been widely believed
as one of the main mechanisms underlyinge sQTLs, we are unaware of any direct evidence of this for
tissues. It is well known that sQTLs are tissue specific [3]. Moreover, Schmitt et al. [23] recently identified hotspots
of local chromatin interactions from Hi-C data, called frequently interacting regions (FIREs). FIREs are
bins that frequently interact with nearby regions $<200$Kb, and they display strong tissue specificity. It is
unclear how much overlap exists between tissue-specific
FIREs and tissue-specific sQTLs.



\subsection{Related Work}

Joint analyses of multi-tissue Hi-C and eQTL data demonstrate close spatial proximity
between eQTLs and their target genes by~\cite{yu2019} and~\cite{duggal2013} discuss work similar to ours. 

ASpedia: Alternative Splicing Encyclopedia of Human

\section{Experiments and Datasets}

Are SQTLs occur often near TAD boundaries?

We obtain sQTLs for cancer cells from CancerSplicingQTL database~\cite{tian2018}~\url{http://www.cancersplicingqtl-hust.com/#/sqtls}, which
compiles genome-wide sQTL data from publications~\cite{}. We intersect gene names in this
database with gene names or IDs in the Ensembl database~\cite{yates2019} and select genes that are associated with a
unique range in the Ensembl database to produce a collection $S$ of sQTLs.

\begin{itemize}
\item Prostate cancer: A high-resolution 3D epigenomic map reveals insights into the creation
of the prostate cancer transcriptome. \url{https://www.ncbi.nlm.nih.gov/geo/query/acc.cgi?acc=GSE118629}
In this paper, normal prostate (RWPE1) and prostate cancer cells (C42B
and 22Rv1). sQTL is in CancerSplicingQTL database under PRAD~(Prostate Adenocarcinoma)~\cite{tian2018}.
\vspace{0.1cm}
\item LAML~(Acute Myeloid Leukemia): is of K562 cell type. Hi-C data for K562 cell is obtained in~\cite{rao2014}, sQTL is in CancerSplicingQTL database~\cite{tian2018}.
\vspace{0.1cm}
\item CML~(Chronic Myeloid Leukemia): is of KBM7 cell type. Hi-C data
  for K562 cell is obtained in~\cite{rao2014}, sQTL is in
  CancerSplicingQTL database~\cite{tian2018}.
\vspace{0.1cm}
\item Multiple Myeloma (MM): $3$D genome of multiple myeloma reveals
  spatial genome disorganization associated with copy number
  variations. GEO GSE87585. Could not find sQTL database.
\vspace{0.1cm}
\item Breast Cancer~(BRCA): \textit{Information will be added!}
\vspace{0.1cm}
\item Glioblastoma~(GBM): \textit{Information will be added!}
\vspace{0.1cm}
\item Dorsolateral Prefrontal Cortex~(DLPFC): sQTL is available in
  Supplementary files of Genome-wide identification of splicing QTLs
  in the human brain and their enrichment among
  schizophrenia-associated loci paper~\cite{takata2017}. Hi-C data is
  available in~\cite{schmitt2016}.
  \vspace{0.1cm}
\item Hippocampus: Hi-C data is available in A Compendium of Chromatin Contact Maps Reveals Spatially Active Regions in the Human Genome~\cite{schmitt2016}.
\vspace{0.1cm}
\item Some drugs also target $3$D genome organization as
in~\cite{kantidze2019} on HT1080
cells~(fibrosarcoma(SARC)). https://www.ncbi.nlm.nih.gov/geo/query/acc.cgi?acc=GSM3304262. Cool extension
files are interaction matrices. sQTLs are available under xxx in
CancerSplicingQTL
\end{itemize}


Some research questions we are interested in answering:

\begin{itemize}
\item Relationship between chromatin interaction frequency and number of sQTLs.
  What about TADs? We implement variety of regression models such as:
1- Negative binomial regression with covariates xxx and yyy.

  Negative binomial regression is implemented in statsmodels package in
Python.

Relationships between gene orthologs as well.

Gene Ontology

 \vspace{0.1cm}
\item Does sQTLs appear more frequently in TAD boundaries?
\vspace{0.1cm}
\item How is median-survival time is correlated with xxx?
\vspace{0.1cm}
\item How is phenotype impacted?
\vspace{0.1cm}
\item How does splicing type impact?
\vspace{0.1cm}
\item Cis-trans difference
\vspace{0.1cm}
\item ddd
\end{itemize}

Negative binomial regression is implemented in statsmodels package in
Python.


%\inputencoding{latin2}
\bibliographystyle{plain}
\bibliography{qtl}
%\inputencoding{utf8}

\end{document}